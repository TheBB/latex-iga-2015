\documentclass[12pt,a4paper,notitlepage,english]{article}

\usepackage[utf8]{inputenc}
\usepackage[T1]{fontenc}
\usepackage{babel}

\usepackage{fullpage}

\begin{document}

\title{A mixed-order isogeometry solver for poroelasticity problems}
\author{
  E. Fonn
  \thanks{\texttt{eivind.fonn@sintef.no}}
  \thanks{Applied Mathematics, SINTEF ICT, Strindvegen 4, 7034 Trondheim, Norway}
  \and
  Y. W. Bekele
  \thanks{Department of Civil and Transport Engineering, NTNU}
  \and
  T. Kvamsdal
  \thanks{Department of Mathematical Sciences, NTNU}
  \and
  A. M. Kvarving \footnotemark[2]
  \and
  S. Nordal \footnotemark[3]
}

\pagenumbering{gobble}
\maketitle

\abstract{
  Isogeometric methods for solving poroelasticity problems was first introduced in 2013 by Irzal
  et. al. \cite{Irzal2013ife}, which highlighted the advantages of smooth basis functions using
  equal polynomial orders for both displacement and pore pressure.  We present here a mixed-order
  formulation where the pore pressure has reduced order, with the aim of obtaining a method which is
  more stable for small timesteps, and which can avoid the pressure oscillations plaguing more
  conventional equal-order methods.  Results from one- and two-dimensional consolidation problems
  are shown.
}

\bibliography{common/references}
\bibliographystyle{naturemag}

\end{document}
